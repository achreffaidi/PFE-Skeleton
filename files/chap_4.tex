\chapter{Quality Assurance}

\label{chap_4}


\section*{Introduction}
\addcontentsline{toc}{section}{Introduction} 
\qquad Monitoring and alerting solutions enable IT professionals to read and analyze infrastructure data and logs to obtain insight into the performance of applications and systems. In the case of incidents, they enable system administrators to remedy the situation and avert additional problems immediately. They are also valuable for analyzing application performance and, in some situations, reducing costs. 

This chapter will describe some of the monitoring services we experimented with and the end outcome. We will also demonstrate the Load-Testing project we used to assess the product performance for various setups.

\newpage

\section{Monitoring and Alerts}
\qquad This section will begin by defining what we expect from the Monitoring and Alert system. Then we'll go through the differences between the two monitoring services we tried out and why we chose one of them.

\subsection{Requirements}

placeholder paragraph, placeholder paragraph, placeholder paragraph, placeholder paragraph, placeholder paragraph, placeholder paragraph,

\subsection{Tool-a}



\subsection{Tool-b}


\subsection{Comparison}

The following table shows some of the differences between Tool-a and Tool-b: 


\section{Dashboard and alerts deployment}
\qquad 

This section will present how the dashboard is being developed, maintained and deployed. 

\subsection{Infrastructure design}


\subsection{Usage of Terraform}

\qquad
placeholder paragraph, placeholder paragraph, placeholder paragraph, placeholder paragraph, placeholder paragraph, placeholder paragraph,

\subsection{Results}
The following screenshot\ref{fig:monitoring-dashboard} shows the end result: 

\begin{figure}[!h]
    \centering
    \includegraphics[scale=0.73]{figures/placeholder.png}
    \caption{Monitoring Dashboard}
    \label{fig:monitoring-dashboard}
\end{figure}

\section{Load Testing}
\qquad As it is vital to ensure high performance in our production environment, we decided to create a Load-testing project that can help us decide which would be the optimal configuration for our infrastructure. This section will list the load-test requirements, the testing framework, the load-testing infrastructure, and the final report. 

\subsection{Requirements}


placeholder paragraph, placeholder paragraph, placeholder paragraph, placeholder paragraph, placeholder paragraph, placeholder paragraph,

\subsection{Tool-a}

\begin{figure}[!h]
    \centering
    \includegraphics[scale=0.75]{figures/placeholder.png}
    \caption{Tool Logo}
\end{figure}


\subsection{Infrastructure for Load Testing}




\subsection{Load Testing execution}

A normal execution of a test would be as follow: 
\begin{enumerate}
\item aaa
\item aaa
\item aaa
\item aaa
\item aaaa
\item aaaa
\item 
\end{enumerate}

\subsection{Reports and Insights}

placeholder paragraph, placeholder paragraph, placeholder paragraph, placeholder paragraph, placeholder paragraph, placeholder paragraph.

\section*{Conclusion}
\addcontentsline{toc}{section}{Conclusion} 
\qquad

In this chapter, we had the opportunity to go through the different measures we have in place to ensure that our system is running smoothly and error-free. The next chapter will cover how we managed to automate the deployment of both the product and its monitoring \& alerts system. 
