\chapter{General context}
\label{chap_1}
\section*{Introduction}
\addcontentsline{toc}{section}{Introduction} 
\label{section_intro}
\qquad The first parts of this chapter will be devoted to introducing the host company where this project was conducted, the partners we worked with, and the data-exchange solution we're going to deploy. We will then describe the motivation behind this project and the proposed solution. Finally, we will conclude by identifying the work methodology and the project timeline.

\newpage

\section{Host organization}
\qquad The work presented in this report was conducted as part of the DevOps team in company-a. This section will be dedicated to describing the company and its field of expertise and distinguished partners.

\subsection{Description of the company}
\qquad 
placeholder paragraph, placeholder paragraph, placeholder paragraph, placeholder paragraph, placeholder paragraph, placeholder paragraph,
\cite{company}

\subsection{Fields of expertise}
\qquad 

placeholder paragraph, placeholder paragraph, placeholder paragraph, placeholder paragraph, placeholder paragraph, placeholder paragraph,

\begin{itemize}[label=\textbullet]
    \item \textbf{Cloud and DevOps:} placeholder paragraph, placeholder paragraph, placeholder paragraph, placeholder paragraph, placeholder paragraph, placeholder paragraph,

    \item \textbf{Data Science and Machine Learning: } placeholder paragraph, placeholder paragraph, placeholder paragraph, placeholder paragraph, placeholder paragraph, placeholder paragraph,

    \item \textbf{Full Stack Web Development:} placeholder paragraph, placeholder paragraph, placeholder paragraph, placeholder paragraph, placeholder paragraph, placeholder paragraph.
 
    \item \textbf{Mobile development:} placeholder paragraph, placeholder paragraph, placeholder paragraph, placeholder paragraph, placeholder paragraph, placeholder paragraph.
\end{itemize}

\subsection{Partners}
\qquad 

placeholder paragraph, placeholder paragraph, placeholder paragraph, placeholder paragraph, placeholder paragraph, placeholder paragraph,
:

\begin{itemize}[label=\textbullet]
    \item \textbf{abc1:} placeholder paragraph, placeholder paragraph, placeholder paragraph, placeholder paragraph, placeholder paragraph, placeholder paragraph.
    \item \textbf{abc2: } placeholder paragraph, placeholder paragraph, placeholder paragraph, placeholder paragraph, placeholder paragraph, placeholder paragraph.
    \item \textbf{abc3:} placeholder paragraph, placeholder paragraph, placeholder paragraph, placeholder paragraph, placeholder paragraph, placeholder paragraph.
    \item \textbf{abc4:} placeholder paragraph, placeholder paragraph, placeholder paragraph, placeholder paragraph, placeholder paragraph, placeholder paragraph.
\end{itemize}


\section{Project actors}
\qquad placeholder paragraph, placeholder paragraph, placeholder paragraph, placeholder paragraph, placeholder paragraph, placeholder paragraph.

\subsection{Partnar-a}
\qquad placeholder paragraph, placeholder paragraph, placeholder paragraph, placeholder paragraph, placeholder paragraph, placeholder paragraph,

\subsection{Partnar-b}
\qquad placeholder paragraph, placeholder paragraph, placeholder paragraph, placeholder paragraph, placeholder paragraph, placeholder paragraph,

\subsection{Partnar-c}
\qquad placeholder paragraph, placeholder paragraph, placeholder paragraph, placeholder paragraph, placeholder paragraph, placeholder paragraph,


\section{Motivations and problem statement}

placeholder paragraph, placeholder paragraph, placeholder paragraph, placeholder paragraph, placeholder paragraph, placeholder paragraph.

\subsection{Motivation}

\quad placeholder paragraph, placeholder paragraph, placeholder paragraph, placeholder paragraph, placeholder paragraph, placeholder paragraph.



\subsection{Problem statement}

\quad placeholder paragraph, placeholder paragraph, placeholder paragraph, placeholder paragraph, placeholder paragraph, placeholder paragraph.
\section{Proposed solution}

\quad placeholder paragraph, placeholder paragraph, placeholder paragraph, placeholder paragraph, placeholder paragraph, placeholder paragraph,

\section{Project framework and methodology}

This section will introduce the Agile scrum methodology and illustrate why our team chose it to implement our project!

\subsection{Agile methodology}
\quad placeholder paragraph, placeholder paragraph, placeholder paragraph, placeholder paragraph, placeholder paragraph, placeholder paragraph:

\begin{enumerate}
\item Individuals and interactions over processes and tools.
\item Working software over comprehensive documentation.
\item Customer collaboration over contract negotiation.
\item Responding to change over following a plan.
\end{enumerate}


Our team uses a combination of two agile frameworks in a way that matches our requirements and team size: 

\begin{itemize}[label=\textbullet]
\item \textbf{SCRUM framework: } Is a sort of agile approach that breaks projects down into manageable parts known as "sprints". It also specifies roles and responsibilities within the SCRUM team. 

\item \textbf{Extreme programming framework: } It creates a minimal, yet effective environment that allows teams to become highly productive. XP emphasizes the importance of pair-programming sessions which is something that influenced a lot our daily work routine.  
\end{itemize}


\subsection{Team and roles }

Our team structure contains the following roles as it's recommended by SCRUM:

\begin{itemize}[label=\textbullet]
    \item \textbf{Developers:} A developer on a scrum team is anyone on the team who is delivering work, including team members who do not work in software development.
    \item \textbf{Product Owner:} Maintains the product's vision and prioritizes the product backlog.
    \item \textbf{Scrum Master:} Assists the team in making the best use of scrum to produce the product.
\end{itemize} 



The following table\ref{tab:team-structure} shows the structure of our scrum team:

\begin{table}[h]
\centering
\begin{tabular}{|c|l|ll}
\cline{1-2}
\cellcolor[HTML]{DAE8FC}{\color[HTML]{000000} Role} & \multicolumn{1}{c|}{\cellcolor[HTML]{DAE8FC}{\color[HTML]{000000} Members}}                                                            &  &  \\ \cline{1-2}
Product Owner                                       & desciption                                                                           &  &  \\ \cline{1-2}
SCRUM Master                                        & descrition                                                                                  &  &  \\ \cline{1-2}
Developers                                          & \begin{tabular}[c]{@{}l@{}}8 Software Engineers from company-a: \\  - 3 DevOps Engineers \\  - 5 Web Development Engineers\end{tabular} &  &  \\ \cline{1-2}
\end{tabular}
\caption{Scrum team structure}
\label{tab:team-structure}
\end{table}


\subsection{The SCRUM events}

\qquad placeholder paragraph, placeholder paragraph, placeholder paragraph, placeholder paragraph, placeholder paragraph, placeholder paragraph.

\newpage

\section{Project timeline}

My contribution to the project will cover the first eight sprints as follows\ref{tab:sprints}: 

\begin{table}[htbp]
\centering
\begin{tabular}{|c|c|c|l|}
\hline
\rowcolor[HTML]{DAE8FC} 
{\color[HTML]{000000} Sprint} & {\color[HTML]{000000} Duration} & \multicolumn{1}{l|}{\cellcolor[HTML]{DAE8FC}\begin{tabular}[c]{@{}l@{}}Working \\ days\end{tabular}} & \multicolumn{1}{c|}{\cellcolor[HTML]{DAE8FC}Tasks}                                                                                                          \\ \hline
0                             & 26/02  - 02/03                  & 5                                                                                                    & \begin{tabular}[c]{@{}l@{}}- On-boarding\\ - aaa\end{tabular}                                               \\ \hline
1                             & 02/03 - 13/03                   & 10                                                                                                   & \begin{tabular}[c]{@{}l@{}}- aaa\\ - aaa \\ \end{tabular}                                                                    \\ \hline
2                             & 16/03 - 27/03                   & 10                                                                                                   & \begin{tabular}[c]{@{}l@{}}  - aaa\\ - aaa   \\    \end{tabular}                                                                                          \\ \hline
3                             & 30/03 - 10/04                   & 10                                                                                                   & \begin{tabular}[c]{@{}l@{}} - aaaa       \\- aaa\\ - aaa\end{tabular}                                                        \\ \hline
4                             & 13/04 - 24/04                   & 10                                                                                                   & \begin{tabular}[c]{@{}l@{}}- aaa\\ - aaa\\ - aaa\end{tabular}                                      \\ \hline
5                             & 27/04 - 08/05                   & 10                                                                                                   & \begin{tabular}[c]{@{}l@{}}- aaa\\ - aaaa\\ - aaaa\end{tabular} \\ \hline
6                             & 11/05 - 29/05                   & 15                                                                                                   & \begin{tabular}[c]{@{}l@{}}- aaa\\ - aaa\\ - aaa\end{tabular}             \\ \hline
7                             & 01/06 - 11/06                   & 10                                                                                                   & \begin{tabular}[c]{@{}l@{}}- aaaa\\ - aaa\end{tabular}                                                               \\ \hline
\end{tabular}
\caption{Sprints} 
\label{tab:sprints}
\end{table}

\section*{Conclusion}

\addcontentsline{toc}{section}{Conclusion} 

\qquad This chapter was dedicated to introducing the host organization and its partners in the first place. Then we presented the problem statement and the proposed solution. It also allowed us to demonstrate the work approach we will use throughout our project. 

The following chapter will go over some of the theoretical aspects of DevOps and Cloud Engineering that are required for the project's implementation.